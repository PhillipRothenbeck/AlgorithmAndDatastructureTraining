\documentclass[runningheads]{llncs}
\usepackage[T1]{fontenc}
\usepackage{graphicx}

\begin{document}
\title{Exam Assignments}

\author{Phillip Rothenbeck}
\institute{Training for Programming Contests and Interviews}

\maketitle

\section*{Assignment 1}

\subsection*{What is the time complexity of the following code ... ?}%

\subsubsection*{Code 2} \label{1:1:code2}
\begin{verbatim}
    void print_pairs(const vector<int> &vec) {
        for (size_t i = 0; i < vec.size(); i++) {
            for (size_t j = 0; j < vec.size(); j++) {
                cout << "(" << vec[i] << ", " << vec[j] << ") ";
            }
        }
    }
    
\end{verbatim}
The code prints out all pairs, that are possible in the set of numbers appearing in the input vector. To achive this
the program iterates for each number of the vector through the vector, resulting in the runtime complexity of
$\mathcal{O}(n^2)$

\subsubsection*{Code 6}
\begin{verbatim}
    void reverse(vector<int> &vec) {
        for (size_t i = 0; i < vec.size() / 2; ++i) {
            size_t other = vec.size() - i - 1;
            int temp = vec[i];
            vec[i] = vec[other];
            vec[other] = temp;
        }
    }
\end{verbatim}
The code reverses the order of the elements in the entered vector. In each iteration two elements are exchanged,
which means only half of the vector needs to be traversed. The runtime is $\frac{n}{2} \in \mathcal{O}(n)$

\pagebreak
\subsubsection*{Code 8}
\begin{verbatim}
    uint64_t factorial(uint64_t n) {
        if (n == 0) {
            return 1;
        } else {
            return n * factorial(n - 1);
        }
    }
\end{verbatim}
Since only one recursive call is made per function call the recursion tree is only one line, which terminates when
$n = 0$. This happens after $n$ recursive call and concludes in a time complexity of $\mathcal{O}(n)$


\subsection*{For each of the following time complexities: $\mathcal{O}(1)$, $\mathcal{O}(log n)$, $\mathcal{O}(n)$,
    $\mathcal{O}(n log n)$ and $\mathcal{O}(n^2)$ give an algorithmic example.}

\subsubsection*{$\mathcal{O}(1)$}
\begin{verbatim}
    int add(int a, int b) {
        return a + b; 
    }
\end{verbatim}
The function $add$ adds its inputs $a$ and $b$ together. Since this is done in constant time, the time complexity is
$\mathcal{O}(1)$

\subsubsection*{$\mathcal{O}(log n)$}
Binary search has a time complexity of $\mathcal{O}(log n)$. The input array is already sorted. Using this information
the algorithm compares the searched number with only the median of the array, which splits the array in two halves.
Binary search recursively repeats this for the half, that could contain the searched number until it gets an empty
input. The input array is a preorder traversal of a binary tree, which has a height of $log n$. The algorithm wanders
only on one path between root and leaf, which concludes in the runtime of $\mathcal{O}(log n)$.

\subsubsection*{$\mathcal{O}(n)$}
\begin{verbatim}
    int sum(int n, int *arr) {
        int sum = 0;
        for (int i = 0; i < n; i++) {
            sum += arr[i];
        }
        return sum;
    }
\end{verbatim}
For adding all numbers of an array together, an algorithm has to add each number to the sum. Resulting in $\mathcal{O}(n)$.

\subsubsection*{$\mathcal{O}(n log n)$}
The sorting algorithm Merge-Sort sorts an array of n numbers in $\mathcal{O}(n log n)$ time. When sorting the resulting
recursion tree is binary and has the height of $log n$ with $n$ leafs. It has to go through every path from the root
to each leaf (and combine each leaf to a sorted list afterwards in $log n$). This means it has a runtime of
$(log n + log n) \cdot n = 2 log n \cdot n \in \mathcal{O}(n log n)$

\subsubsection*{$\mathcal{O}(n^2)$}
\begin{verbatim}
    int sum(int n, int **m) {
        int sum = 0;
        for (int i = 0; i < n; i++) {
            for (int j = 0; j < n; j++) {
                sum += m[i][j];
            }
        }
        return sum;
    }
\end{verbatim}
This algorithm sums up all elements of an 2D array with n lines and n columns. For that it has to traverse all $n^2$
elements.

\subsection*{Outline the idea of how to retrospectively compute in $\mathcal{O}(n)$ the maximum profit when buying and then selling
    a stock once.}

The algorithm traverses the array of prices ones it saves the minimal price, that it has come across so far and the
corresponding profit, that could have been made. For each price it calculates the profit from the saved minimal price
and stores it, if it is bigger than the currently stored maximal profit it replaces that number. In addition, if the
current price is smaller than the lowest stored price it will replace the current lowest price. This way the array
needs to be traversed only once to calculate the highest profit.

\subsection*{How would you check if string b is a permutation of string a.}

I would check for both strings how often a letter is appearing in it. If the same letters appear in the same amount
in both strings, they are permutations of each other.

\section*{Assignment 2}

\subsection*{How can we exploit bit operations for operations with sets?}

A set can be converted into a bitset. The bitset is a binary number, in which a certain digit shows if an element exists
in the set(1) or not(0). This can be exploited to do fast comparative operations on two sets like $and$ (intersection
of two sets), $or$ (union of two sets) and $xor$ (symmetric difference of two sets). Since this happens with only one
instruction it is significantly faster than when working with other set implementations.

\subsection*{Explain the difference between arithmetic and logical right shifts for negative signed integers.}
When using the logical right shift, uses 0s to fill the "gap" that is created on the left through shifting the
numbers to right. This is working for unsigned numbers, but if the first bit is used as a sign bit this could
potentially change the sign of the number. To prevent this problem the arithmetic right shift fills the "gap"
with the same value as the sign bit.

\subsection*{Outline an algorithm to count the number of bits that are set to 1 in an integer. (no bitset or built-in functions)}
For this I would use a bit cursor, that always has exactly one bit set to 1 and is of the same length as the input
number. The algorithm iterates over each bit and moves the 1 in the cursor to the looked upon digit. The intersection of
the number and the cursor is formed. If it is greater than 0 a counter is incremented by one.

\begin{verbatim}
    uint_8 count_1_bits(uint_32 n) {
        uint_8 counter = 0;
        for (int i = 0; i < 32; i++) {
            cursor = 1u << i;
            if((cursor & n) > 0)  {
                counter++;
            }
        }
        return counter;
    }
\end{verbatim}

\subsection*{Outline an algorithm to reverse the bits in an unsigned integer (swap bits at position i and n - i).}
We have an index i that increments per iteration by one until it has reached the middle of the binary number. Now
we take to cursors $first\_digit$ with a 1 at the $i$'th digit and $second\_digit$ with a 1 at the $n-i$'th. The
$first\_digit$ cursor is shifted n-i times to the right and the $second\_digit$ cursor is shifted n-i times to the
left. Lastly a union of the cursors and the output number is formed. The swapping is performed through the bit shifting.
If the input number has an odd count of digits i runs to $\frac{bit\_length(n - 1)}{2}$

\subsection*{Imagine you have to reorder integers in an array so that the odd entries appear first. How is this problem related to Quicksort’s partitioning function?}
We introduce an index ($first\_even$) that sits at the position, where the first even number sits and all numbers to
the left are odd. The other index($i$) is running across all numbers of the array. Every time $i$ comes across an odd
number it swaps it with the number at the position of $first\_even$ and increments this index by one. The output array
has all its odd numbers appearing first.

\section*{Assignment 3}

\subsection*{Outline an algorithm to convert a string to an unsigned integer (no negative numbers).}
The algorithm subtracts 48 from each character in the string and multiplies it by the corresponding power
of ten before adding it to the output number.

\begin{verbatim}
    Input: String s and its length n
    digit = 10 ^ (n - 1)
    number = 0
    i = 1
    repeat n times:
        c = i'th character of s
        x = (get ASCII value of c) - 48
        number = number + x * digit
        i = i + 1
        digit = digit / 10
    Output: number
\end{verbatim}

\subsection*{Outline an algorithm to convert an unsigned integer to a string.}
The algorithm starts to allocate a string with a number of characters equal to the number of bits of the integer plus one.
Each digit is extracted, 48 is added, and the resulting ASCII interpretation is stored in the corresponding position
in the string.

\begin{verbatim}
    Input: integer x, stored in b bits
    allocate string s with b + 1 characters and initialize the to ' '
    pow = 10
    rest = 0
    repeat b times with index i from b - 1 to 0:
        digit = (x % pow) - resulting
        s[i] = digit + 48 // results in the ASCII value of the digit
        pow = pow * 10
        rest = rest + digit
    Output: s
\end{verbatim}

\subsection*{Explain the characteristics of the character encodings ASCII, Extended ASCII, UTF-8, UTF-16 and UTF-32.}
ASCII uses 7 bits to encode 128 characters, while Extended ASCII uses 8 bits to encode 256 characters. UTF-8 on the
other hand uses its first bits to declare how many bytes are supposed to be read to get the character. UTF-8 is a
superset of ASCII. UTF-16 can use one or two words of each 16 bits to encode one character, while UTF-32 has a fixed size
of 4 bytes for all characters that it can encode.

\subsection*{What advice can you give to modify strings as quickly as possible?}
To avoid the time-consuming process of copying the entire string, string modifications should be handled in-place.
If copying is unavoidable, minimize the number of times the string is copied. This can be
achieved by marking the positions where modifications are to be made and finalizing them during the copying process.

\subsection*{Why are hardcoded regex expressions often much faster?}
Before using a regular expression it has to be compiled into a state machine, which is time-consuming.
If the state machine for a certain regular expression is already known, it is faster to hardcode the
state machine since this eliminates the step of automatically compiling. Additionally, exploiting
certain characteristics of the regular expression can speed up runtime.

\subsection*{Is run-length encoding a good compression method for strings?}
This depends on the string that is supposed to be encoded. For instance the string cannot contain any digits,
also it should not have a high frequency of changing characters. This could result in the encoded string
being longer than the original. For strings that have a lot of longer sequences of alphabetical letters,
run-length encoding may be a good compression method.

\section*{Assignment 4}

\subsection*{How to delete a (non-tail) node from a singly linked list without knowing its predecessor in $\mathcal{O}(1)$ time?}
To achieve this, copy the values of the successor into the node containing the data to be deleted. Then, set the
pointer of this node to the successor of its successor and delete the successor. This reduces the number of nodes
by one and deletes the data of the node to be deleted.

\subsection*{Explain the runner technique using a linked list example.}
The runner technique involves using multiple pointers to iterate through a linked list at different speeds.
This method can be used to find the pivot element that has 1/4 of all nodes to its right. One pointer
(the runner) moves four times for each move the other pointer makes.

\subsection*{If you were to tune the code for speed, how would you implement a stack?}
I would use a \verb|std::vector| to implement a stack. Since this datastructure provides the method that a
stack needs like \verb|push| (as \verb|push_back|), \verb|pop| (as \verb|pop_back|), \verb|empty| and
\verb|top| (as \verb|back|). These functions are efficiently implemented, making them fast enough.

\subsection*{How can we use a std::list in C++ as a queue?}
To implement a queue using the \verb|std::list|, we first need to specify which side of the list should be the entrance point
and which will be the exit. In this example, we use \verb|front| as the entrance and \verb|back| as the exit. To enqueue,
we use the list function \verb|emplace_front| which constructs a new node at the front of the list with the new data. For
dequeuing, we first save the data and then use \verb|pop_back| to delete the note. Finally, we return the data. The
\verb|size| and \verb|empty| data give the information how big our queue is and if it is empty. Thus, all necessary
mechanisms can be implemented using \verb|std::list|.

\subsection*{How does a binary tree and a binary search tree differ?}
A binary tree is a tree in which each node has at most two children. A binary search tree is a type of binary tree that adds
the condition that for every node x, the left subtree only contains nodes that are lower or equal to x, and the right subtree only contains
nodes being greater than x. Therefore, binary search trees are a strict subset of binary trees.

\subsection*{Describe the different types of binary tree traversals.}
There are three types of tree traversals. First is pre-order, meaning that we first process the current node followed by the left
subtree, and then the right subtree. The second is in-order, which processes the left subtree first, then the current node, and finally
the right subtree. Lastly there is post-order, where we start with the left subtree, continue with the right subtree, and finally process
the current node.

\subsection*{How does a trie and a radix tree differ?}
Both a trie and a radix tree hold information for words in a path from the root. However, a trie assigns exactly one node to each character,
while a radix tree is more compressed as it allows for a node to hold more than one letter. As a result, a radix tree requires fewer nodes than
a trie, to store the same words as it merges a path that has no branches into one node.

\section*{Assignment 5}

\subsection*{Describe briefly the heap data structure.}
The heap is a binary tree that has two conditions. Firstly, a child at depth k can only exist if all children at depth of k - 1 exist.
This means that a new 'row' of children can only start, when the previous one is completely filled. Otherwise, a new child will be added from
left to right in the incomplete row. Secondly, each node must be greater than or equal to its children. The time complexity for insertion is
$\mathcal{O}(log n)$ as the heap characteristics must be maintained. Finding the maximum element takes $\mathcal{O}(1)$ time since the element
is always located at the root. Deleting the maximum takes $\mathcal{O}(log n)$ time, while searching for an item has a complexity of
$\mathcal{O}(n)$.

\subsection*{How would you find the k longest words in a data stream? (You cannot back up to read an earlier value.)}
Utilizing a min-heap can accomplish this task. Each word from the stream is added to the heap. The heap size is checked when a word is added to
ensure it does not exceed k. If the size exceeds k, the shortest word in the heap is deleted. Once the whole stream has been processed, the heap
will only contain the k longest words.

\subsection*{How would you address the problem of merging multiple sorted files that are too large for RAM}
To accomplish this task, a min-heap can be defined with only the smallest number of each file, resulting in a heap of the same size as the
number of files to be merged. For merging, I iteratively would take the minimum from the min-heap, delete it and add it onto the resulting file.
Then, I would push the next smallest number from the file that the minimum originated from. Thus, I would only have to have loaded the smallest number
of each file and for each iteration load from the next number, which has a time complexity of $\mathcal{O}(1)$.

\subsection*{What are sorting networks?}
Sorting networks are data-oblivious sorting algorithms for arrays of a set size. Each network is constructed using out compare-and-exchange
(COEX) modules, which compare two input numbers and exchange them if the condition is met. To be able to sort with these modules they have to be
put into a specific order of comparing different elements of the array. This means that a sorting net can be hardcoded.

\subsection*{Why sorting may speed up set operations?} %not checked
Sorting reduces the amount of work that it takes to find a particular item in a set. Searching is a large part of set operations (when not
represented as a bitset), since the availability of the elements of one set must be checked in another set.

\subsection*{What is a minimal perfect hash?}

\begin{thebibliography}{8}
    %\bibitem{ref_url_pcb}

\end{thebibliography}
\end{document}